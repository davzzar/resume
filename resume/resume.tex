%!TEX TS-program = xelatex
%!TEX encoding = UTF-8 Unicode
% Awesome CV LaTeX Template for CV/Resume
%
% This template has been downloaded from:
% https://github.com/posquit0/Awesome-CV
%
% Author:
% Claud D. Park <posquit0.bj@gmail.com>
% http://www.posquit0.com
%
% Template license:
% CC BY-SA 4.0 (https://creativecommons.org/licenses/by-sa/4.0/)
%


%-------------------------------------------------------------------------------
% CONFIGURATIONS
%-------------------------------------------------------------------------------
% A4 paper size by default, use 'letterpaper' for US letter
\documentclass[11pt, a4paper]{awesome-cv}

% Configure page margins with geometry
\geometry{left=1.4cm, top=.8cm, right=1.4cm, bottom=1.8cm, footskip=.5cm}

% Specify the location of the included fonts
\fontdir[fonts/]

% Color for highlights
% Awesome Colors: awesome-emerald, awesome-skyblue, awesome-red, awesome-pink, awesome-orange
%                 awesome-nephritis, awesome-concrete, awesome-darknight
\colorlet{awesome}{awesome-orange}
% Uncomment if you would like to specify your own color
% \definecolor{awesome}{HTML}{CA63A8}

% Colors for text
% Uncomment if you would like to specify your own color
% \definecolor{darktext}{HTML}{414141}
% \definecolor{text}{HTML}{333333}
% \definecolor{graytext}{HTML}{5D5D5D}
% \definecolor{lighttext}{HTML}{999999}

% Set false if you don't want to highlight section with awesome color
\setbool{acvSectionColorHighlight}{false}

% If you would like to change the social information separator from a pipe (|) to something else
\renewcommand{\acvHeaderSocialSep}{\quad\textbar\quad}


%-------------------------------------------------------------------------------
%	PERSONAL INFORMATION
%	Comment any of the lines below if they are not required
%-------------------------------------------------------------------------------
% Available options: circle|rectangle,edge/noedge,left/right
% \photo[rectangle,edge,right]{./examples/profile}
\name{David Simon}{Tetruashvili}
\position{Undergraduate Computer Science Student{\enskip$\cdot$\enskip}Tinkerer}
\address{Lambeth, London, United Kingdom.{\enskip$\cdot$\enskip}Berlin, Germany.}

\mobile{(+44) (0) 786–407–5561{\enskip$\cdot$\enskip}(+49) (0) 172–323–0427}
\email{david.tetruashvili@kcl.ac.uk}
%\homepage{}
\github{davzzar}
\linkedin{DavidSTetruashvli}
% \gitlab{gitlab-id}
% \stackoverflow{SO-id}{SO-name}
% \twitter{@twit}
% \skype{skype-id}
% \reddit{reddit-id}
% \medium{madium-id}
% \googlescholar{googlescholar-id}{name-to-display}
%% \firstname and \lastname will be used
% \googlescholar{googlescholar-id}{}
% \extrainfo{extra informations}

\quote{``The best oportunity comes when you find you need somethig that does not exists yet."}


%-------------------------------------------------------------------------------
% better undeline command by alexwlchan (https://alexwlchan.net/2017/10/latex-underlines/)
\usepackage{contour}
\usepackage{ulem}

\renewcommand{\ULdepth}{1.8pt}
\contourlength{0.8pt}

\newcommand{\underl}[1]{%
  \uline{\phantom{#1}}%
  \llap{\contour{white}{#1}}%
}
%-------------------------------------------------------------------------------

\usepackage{hyperref}

\begin{document}

% Print the header with above personal informations
% Give optional argument to change alignment(C: center, L: left, R: right)
\makecvheader[C]

% Print the footer with 3 arguments(<left>, <center>, <right>)
% Leave any of these blank if they are not needed
\makecvfooter
  {\today}
    {David S. Tetruashvli~~~·~~~Résumé}
  {\thepage}


%-------------------------------------------------------------------------------
%	CV/RESUME CONTENT
%	Each section is imported separately, open each file in turn to modify content
%-------------------------------------------------------------------------------
%-------------------------------------------------------------------------------
%	SECTION TITLE
%-------------------------------------------------------------------------------
\cvsection{Profile}


%-------------------------------------------------------------------------------
%	CONTENT
%-------------------------------------------------------------------------------
\begin{cvparagraph}

%---------------------------------------------------------
I am a striving trilingual Computer Science bachelor student with an international background to specialize in Artificial Intelligence at King’s College London. Despite my Bachelor specialization, I am keen to pursue the study of Information Security which I have begun at KCL. In addition, I am interested to open myself other diverse branches of computer science at the postgraduate level with aspirations of doctoral research thereafter. I follow the CTF hacking competition scene as a tool for education as well as self-expression. Currently, I am working on my final year project in non-monotonic reasoning under the supervision of Dr Odinaldo Rodrigues.
\end{cvparagraph}

%-------------------------------------------------------------------------------
%	SECTION TITLE
%-------------------------------------------------------------------------------
\cvsection{Skills}


%-------------------------------------------------------------------------------
%	CONTENT
%-------------------------------------------------------------------------------
\begin{cvskills}

%---------------------------------------------------------
\cvskill
  {Languages} % Category
  {English, German, Russian} % Skills

%---------------------------------------------------------
  \cvskill
    {Programming} % Category
    {Python, C/C++, Java, Scala, NodeJS/JavaScript, PHP, MySQL, NoSQL (Firebase)} % Skills

%---------------------------------------------------------
  \cvskill
    {Tools} % Category
    {(Arch) Linux, Zsh/Bash, Git \& GitHub, Digital Ocean} % Skills

%---------------------------------------------------------
  \cvskill
    {Misc.} % Category
    {Video Editing (Adobe Premier Pro), Arduino \& basic electronics} % Skills

%---------------------------------------------------------
\end{cvskills}

%-------------------------------------------------------------------------------
%	SECTION TITLE
%-------------------------------------------------------------------------------
\cvsection{Education}


%-------------------------------------------------------------------------------
%	CONTENT
%-------------------------------------------------------------------------------
\begin{cventries}

    %---------------------------------------------------------
    \cventry
    {Bachelor of Science in Computer Science (with Artificial Intelligence) (Hons.)} % Degree
    {King’s College London --- University of London} % Institution
    {London, United Kingdom} % Location
    {Sep. 2017 - Present} % Date(s)
    {
        \begin{cvitems} % Description(s) bullet points
            \item {\textbf{Current GPA: 89\%}}
            \item {Expected first-class degree.}
            \item {Relevant modules include:}
            \begin{flushleft}
                \underl{Software Engineering Group Projrct} \textit{[Major Project (``Prep."): \textbf{91\%}, Written Examination: \textbf{92\% (Top mark in class)}, Overall: \textbf{87\%}]},
                \underl{Introduction to Artificial Intelligence} \textit{[Coursework \#1: \textbf{85\%}, Coursework \#2: \textbf{90\%}, Overall: \textbf{84\%}]},\\
                \underl{Practical Experiences of Programming (C++/Scala)} \textit{[\textbf{91\%}]}, \underl{Programming Practice and Applications (Java)} \textit{[\textbf{96\%}]},\\
                \underl{Internet Systems} \textit{[\textbf{97\% (Top mark in class)}]}, \underl{Database Systems (MySQL)} \textit{[\textbf{90\%}]},\\
                with seven other modules with final grades between \textbf{82\%} and \textbf{90\%}.
            \end{flushleft}
        \end{cvitems}
    }

    %---------------------------------------------------------
    \cventry
    {International Baccalaureate Diploma} % Degree
    {Berlin International School} % Institution
    {Berlin, Germany} % Location
    {Sep. 2015 - May 2017} % Date(s)
    {
        \begin{cvitems} % Description(s) bullet points
            \item {\textbf{Overall 42 points out of possible 45}, including 6/7 points in Higher Level Mathematics.}
            \item {Extended Essay in Computer Science on \textit{Evaluation of Texture Filtering Algorithms applied in VR HUDs}:\\
            \textbf{33/36 points, Grade A}.}
        \end{cvitems}
    }

    %---------------------------------------------------------
    \cventry
    {International-GCSE} % Degree
    {} % Institution
    {} % Location
    {Sep. 2015 - May 2017} % Date(s)
    {
        \begin{cvitems} % Description(s) bullet points
            \item {10 IGCSEs with eight being A* or A, including A*s in Mathematics, Physics, and ICT.}
        \end{cvitems}
    }

    %---------------------------------------------------------
\end{cventries}

%-------------------------------------------------------------------------------
%	SECTION TITLE
%-------------------------------------------------------------------------------
\cvsection{Experience}


%-------------------------------------------------------------------------------
%	CONTENT
%-------------------------------------------------------------------------------
\begin{cventries}

%---------------------------------------------------------
  \cventry
    {Mobile App Developer \& Database Engineer} % Job title
    {Guy's and St. Thomas’ Hospital NHS Trust} % Organization
    {London, United Kingdom} % Location
    {Feb. --- Mar. 2019} % Date(s)
    {
      \begin{cvitems} % Description(s) of tasks/responsibilities
        \item {Developed a medical Doctor-Patient communication and test compliance mobile application (iOS and Android) and admin web app (Vue.js) package called \href{https://davzzar.github.io/prep-page/}{\uline{``Prep.''}} in a team of 8 as the ``Major Project'' during the 2\textsuperscript{nd} year at KCL.}
        \item {Managed and maintained a consistent Git version control history and conventions for the whole duration of the project.}
        \item {Designed and implemented several mobile application modules intended for use by patients aged c. between 40 and 60 years old focusing on accessibility, utilizing Google’s Flutter framework.}
        \item {Prototyped/Designed the UI theme and other UI elements for the mobile application with emphasis on ease of use and readability.}
        \item {Designed and implemented the back-end database structure used for the app with hard emphasis on privacy, information leak prevention, and enterprise scalability using Google’s NoSQL Firebase database.}
        \item {Created a \href{https://drive.google.com/file/d/1Pu6NSpqPXnWcaanUpIu_gd5C1bQnOyQi/preview}{\underl{screencast}} presenting the final application.}
        \item {\textbf{Solution has been shortlisted for funding.}}
      \end{cvitems}
    }

%---------------------------------------------------------
\end{cventries}

\newpage
%-------------------------------------------------------------------------------
%	SECTION TITLE
%-------------------------------------------------------------------------------
\cvsection{Projects}


%-------------------------------------------------------------------------------
%	CONTENT
%-------------------------------------------------------------------------------
\begin{cventries}

%---------------------------------------------------------
\cventry
{Graphics and UI/UX Designer, Frontend Developer} % Job title
{\href{https://github.com/davzzar/clim-AX}{\uline{Clim-EX: The Climate Animation Explorer}}} % Organization
{Oxford, United Kingdom} % Location
{Nov. 2019} % Date(s)
{
    \begin{cvitems}
    \bsep Submission to the \href{http://www.ox.ac.uk/students/news/2019-10-14-oxford-hack-2019}{\uline{Oxford University Hackathon 2019}}.\bsep Strived to create a project that would promote and elevate the discussion and exploration contemporary issues of climate change.\bsep Acted as the graphics designer and co-frontend developer; roles that were crucial to the project's success due to its reliance on visual appeal.\bsep Project was one of the lead contenders to the ``Hacker's Choice'' award at the hackathon.
    \end{cvitems}
}

    %---------------------------------------------------------
    \cventry
    {Designer and Programmer} % Job title
    {An Argumentation Framework Solver} % Organization
    {} % Location
    {Oct. 2019 --- Present} % Date(s)
    {
        \begin{cvitems}
            \bsep Currently designing and developing a \href{https://www.sciencedirect.com/science/article/pii/000437029400041X}{\underl{Dung's Argumentation Framework}} Solver for non-monotonic reasoning as part of my final year individual project (Bachelor's Thesis) at King's College London under the supervision of Dr Odinaldo Rodrigues.\bsep The solver is imaged to use a reduction-based approach to solving both decision and enumeration problems in all four of the original semantics presented in \href{https://www.sciencedirect.com/science/article/pii/000437029400041X}{\underl{(Dung, 1995)}}.
        \end{cvitems}
    }

    %---------------------------------------------------------
    \cventry
    {Designer and Programmer} % Job title
    {Collegiate Information Portal (KCL-info)} % Organization
    {} % Location
    {Oct. 2019 --- Present} % Date(s)
    {
        \begin{cvitems}
            \bsep Currently developing an information retrieval web API for use by staff and students of KCL. The main purpose is to provide easy access to important and up-to-date information around campus and all of its aspects in one unified place.
        \end{cvitems}
    }

    %---------------------------------------------------------
    \cventry
    {Back-end Developer} % Job title
    {KCLge Rental Catalogue Web App} % Organization
    {} % Location
    {Nov. --- Dec. 2018} % Date(s)
    {
        \begin{cvitems}
            \bsep In a team of two, developed back-end for an online-store-style web app using pure PHP and JavaScript to access an external API for game data to be cached in a MySQL database.\bsep Implemented CRUD administrative functionality including multiple account types (guest/user/admin system), game information manipulation and editing and entry, and basic rental functionality such as renting rule/restriction definition and enforcement, black-listing, and game inventory management.
        \end{cvitems}
    }
\end{cventries}

%---------------------------------------------------------

%-------------------------------------------------------------------------------
%	SECTION TITLE
%-------------------------------------------------------------------------------
\cvsection{Extracurricular Activity \& Volunteering}


%-------------------------------------------------------------------------------
%	CONTENT
%-------------------------------------------------------------------------------
\begin{cventries}

  %---------------------------------------------------------
  \cventry
  {Temporary Member/Active Participant} % Job Title
  {KCL Informatics Student-Staff Liaison (Unofficial)} % Organisation
  {London, United Kingdom} % Location
  {Dec. 2019} % Date(s)
  {
    \begin{cvitems} % Description(s) of tasks/responsibilities
      \item {Volunteered time to raise and discuss possible solutions to student concerns with the KCL Artificial Intelligence Planning module with both the administration of the faculty and the module leaders Dr Stefan Edelkamp \& Dr Daniele Magazzeni as a member of the student body.}
    \end{cvitems}
  }

  %---------------------------------------------------------
  \cventry
  {Insight Programme Intern} % Job Title
  {TeachFirst (Placed at Harris Academy Peckham)} % Organisation
  {Peckham, London, United Kingdom} % Location
  {Jun. 2019} % Date(s)
  {
    \begin{cvitems} % Description(s) of tasks/responsibilities
      \item {Underwent crash-course-style pedagogic training focusing on self-development in problem solving, leadership, and innovation through workshops, lectures, and practical exercises.}
      \item {Spent one week at Harris Academy Peckham serving as teaching assistant in the department of mathematics.}
      \item {Planned, prepared resources for, and delivered a mathematics lesson to a year 8 set 1 class.}
      \item {Observing teacher noted \textit{``outstanding use of pre-prepared resources, great rapport built with students, and positive and motivational attitude.''}}
    \end{cvitems}
  }

  %---------------------------------------------------------
  \cventry
  {German and Russian Language Tutor} % Job Title
  {Modern Language Center at King’s College London} % Organisation
  {London, United Kingdom} % Location
  {Apr. 2019 --- Present} % Date(s)
  {
    \begin{cvitems} % Description(s) of tasks/responsibilities
      \item {Volunteered time to run speaking practice sessions with students at the university and persons of the public.}
      \item {Evaluated and gave feedback on the students use of the language and lexicon ultimately to help students express themselves in the language.}
      \item {Helped students build confidence in speaking the language in a safe environment.}
    \end{cvitems}
  }

  %---------------------------------------------------------
  \cventry
  {Workshop Leader} % Job Title
  {Beginner's Programming and Game Development Workshop} % Organisation
  {Berlin, Germany} % Location
  {May. --- Jul. 2017} % Date(s)
  {
    \begin{cvitems} % Description(s) of tasks/responsibilities
      \item {Launched an after school beginner’s programming workshop after learning that the school did not offer adequate education in the topic `due to low demand.'}
      \item {Thought the basics of programming to a class of c. 30 students of mixed ages using Python 3 and the simple 2D game engine Game Maker Studio.}
      \item {This effort was acknowledged in the school’s yearbook that year.}
    \end{cvitems}
  }

  %---------------------------------------------------------
  \cventry
  {Competitor} % Job Title
  {ISMTF Senior Math Competition} % Organisation
  {Vienna, Austria} % Location
  {Mar. 2016} % Date(s)
  {
    \begin{cvitems} % Description(s) of tasks/responsibilities
      \item {Participated in the competition in a team of international students to further my insight into mathematics.}
      \item {Used the experience to further my communication skills and learn from representatives of foreign to me curriculums and teaching methods.}
    \end{cvitems}
  }
\end{cventries}

% \input{sections/honors.tex}
% \input{sections/presentation.tex}
% \input{sections/writing.tex}
% \input{sections/committees.tex}


%-------------------------------------------------------------------------------
\end{document}
